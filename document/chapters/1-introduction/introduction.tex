\chapter{Introduction}\label{introduction}
In the last few decades, there has been a significant technological development in various fields.
Computing technology has advanced rapidly, with the introduction of faster and more powerful processors, and the widespread adoption of cloud computing.\newline
The Internet has also become an integral part of modern life, connecting people and devices across the globe; in addition, the development of faster and more reliable connectivity networks has made it possible to transfer and share data at an even faster rate.\newline
Mobile technology has grown exponentially, thanks to the adoption of smartphones and other mobile devices.\newline
One of the fields that has seen a substantial growth, due to the increasing availability and affordability of devices, sensors and other components, is the Internet of Things (IoT). The term IoT refers to the growing network of physical devices, vehicles, buildings and other items that are embedded with sensors, software and connectivity, which enables these \textit{Things} to collect and share data. These capabilities have the potential to bring significant benefits to society and economy, such as improving public services, increasing efficiency and productivity, and reducing costs.

This technological development is closely connected with the growth of distributed systems. With the increase in computing power and the availability of fast and reliable networks, it has become possible to allow devices and systems to work together and share data, even when they are physically separated.

The general growth just discussed brought new challenges: there is the need of engineering complex software that has to take full advantage of the computational infrastructure available, taking in consideration the unpredictability of changes and the heterogeneity of communication required.

In order to face the new complexities there is the necessity to rethink and renovate the process of software development.

\textit{Aggregate Programming} is a paradigm which aim is to address these requirements. It allows for the easy manipulation of data across devices, making it possible to perform operation on the data of distributed systems, in a simple and efficient manner.\newline
This paradigm has been implemented in different programming languages and platforms: two of them are Protelis and Scafi. 
Both Protelis and Scafi presents strengths, but also weaknesses: in order to address those, a new framework can be a solution.

The work described in this thesis consists in the development of a new aggregate programming framework in Kotlin, which strengths are transparency, minimality and portability. 

\section{Context}
In this section is going to be discussed in details the paradigm of Aggregate Programming, in order to provide some context that is going to be necessary for the following part of this document.

\subsection{Aggregate Programming}


\subsubsection{Domain restriction: alignment}

\section{State of the art}
\subsection{Protelis}
\subsection{Scafi}
\subsection{FCPP}

\section{Motivation and goal}