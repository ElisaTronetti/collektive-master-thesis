\chapter{Conclusion and future work}\label{chapter:conclusion}
During the work done in this thesis, it was necessary to explore in details of a lot of different aspects.

The foundation is the \textbf{aggregate programming} paradigm, which introduced its constructs and the alignment problem.\newline Regarding the alignment, multiple technological alternatives were discussed, and some of them required to dive into the \textbf{metaprogramming}. Specifically, they were analyzed Kotlin annotations and stacktraces, KSP (which stands for Kotlin Symbol Processing) and Kotlin compiler plugins.\newline
As seen in Chapter~\ref{chapter:alignment}, the final solution involved the creation of a Kotlin compiler plugin, that creates a custom stack that is used to align correctly the different devices. The plugin required a deep comprehension of the compiler and the IR used to represent the syntax tree, which was used to manipulate correctly the different elements.\newline
Since the compiler plugin is developed based on an intermediate representation maintained by Kotlin itself, it is expected to not require drastic changes over time.

Collektive project includes the Gradle plugin, the compiler plugin and the DSL, and it lays the foundation of a new aggregate programming tool.

The DSL created achieved the requirements established before its development, which are:
\begin{itemize}
    \item \textbf{Transparency}: the behavior of the DSL is transparent to the final user, it performs exactly what an aggregate paradigm expert would expect. Moreover, the alignment is resolved without needing the developer to explicitly handle it, by using the Kotlin compiler plugin;
    \item \textbf{Minimality}: the DSL exposes only four functions, which are the necessary feature for the aggregate programming:
    \begin{enumerate}
        \item \textbf{neighboring}: to model device-to-neighbors interactions;
        \item \textbf{repeating}: to handle the time evolution;
        \item \textbf{sharing}: to observe the neighbors' field, updated the local values and sharing immediately the updated state;
        \item \textbf{alignedOn}: to take advantage of the alignment in peculiar situations that the compiler plugin typically does not align;
    \end{enumerate}
    \item \textbf{Portability}: since its development is in Kotlin Multiplatform, it is possible to guarantee the compatibility with JVM, JavaScript and Kotlin Native platforms.
\end{itemize}

The validation process verifies the correct system behavior on the different targets, taking in consideration single computation and executions that involved multiple rounds to prove the correct interaction between devices.\newline
The usage of \textbf{Alchemist Simulator}~\cite{alchemist} proves that it is possible to create algorithms, such as the gradient, in a simple way, and that the communication between the devices works as predicted. 

Different improvements can be done in the future:
\begin{enumerate}
    \item It is possible to integrate Collektive into \textbf{Alchemist}~\cite{alchemist}, creating an incarnation that allows to create new algorithms just using the simulator;
    \item Another possible development is the implementation of a library with self-stabilizing functions, which would capture families of strategies used typically to achieve flexible and resilient decentralized behaviors, in order to hide the complexities by using the field calculus constructs~\cite{self_stabilisation_functions};
    \item It would be also necessary in the future to publish the project to a package repository like Maven Central, JCenter, or similar, allowing others to easily download and use Collektive as a dependency in their own projects.
\end{enumerate}